\chapter{Introduction}\label{ch:introduction}
As the presence of computer networks in our day-to-day lives increases, the need for a method to detect attacks or abuse of these networks also increases. A common approach to this problem is the analysis of network log files that contain information about system activities, such as login attempts, transfers of data, etc. The problem is that only a cyber-security expert can conclude whether certain behavior is actually an attack. This also needs to happen very quickly as there tends to be a lot of data in these logs. This calls for a computer system handling this problem as humans simply can't keep up with the amount of data. A system that does this needs to be both fast and good at identifying anomalous behavior, while at the same time being able to adapt to any changes the attackers might make to avoid it. The system should, preferably, also be able to run in real-time, being able to detect any abnormal behavior as it happens. This can be a very important factor in data breaches. The field of Deep Neural Networks (DNNs) seems to present a solution for this problem; it combines both the speed of computers and attempts to mimic the ability of our brains to learn very quickly, which allows it to recognize complex patterns.

%cSpell:words akent
In 2015, a data set\footnote{The data set can be found at https://csr.lanl.gov/data/cyber1/} was published in~\cite{akent-2015-enterprise-data}, containing around 100GB of anonymized event data collected from the US-based Los Alamos National Laboratory's internal network over 58 consecutive days. This data set consists of a number of different types of data: authentication, process, network flow, DNS and red team data. The authentication data is by far the biggest with 1,051,430,459 out of 1,648,275,307 total events. The red team data represents a set of simulated intrusions. This type of data is present to train the system on known intrusions (also known as misuse detection) or to validate the system's findings. However, there is so little red team data (749 actions) that it is not feasible to do this. Since the rest of the data is unlabeled, meaning it is not known whether or not they are actually attacks, the system needs to be trained to recognize users' behavior. The next step is to try to find any deviations from this behavior (also known as anomalies). Because the data consists of a series of actions, sequences of events that are only anomalies when observed together (also known as collective anomalies) might also be in the data set. Collective anomalies would go unnoticed when only reading the data one action at a time. However, a recurrent neural network (RNN), which is particularly good at series of data, is able to find these collective anomalies, making it a perfect fit for this purpose.

The main goal of this thesis is to evaluate the effectiveness of using RNNs for finding anomalies in cyber-security related data, in particular with regards to unsupervised learning (learning on unlabeled data). The approach to this goal is to attempt to find anomalies in the previously mentioned data set, experimenting with different parameters to the neural network and different RNN architectures. Finding anomalies is done by transforming the data set into a vector of features that are then used to train a network to predict the behavior of a user. The network then predicts the next feature vector based on the previous feature vectors. This prediction is then compared to the actual features. If the prediction deviates too much from the mean differences calculated over the training set, these features (and the action they were constructed from) are then classified as anomalies.

This thesis is structured as follows: in Chapter~\ref{ch:related_work} related work is discussed; in Chapter~\ref{ch:rnn} the RNN architecture is explained; in Chapter~\ref{ch:methods} the used methods are described; in Chapter~\ref{ch:experiments} results of the experiments regarding the network's architecture are analyzed; in Chapter~\ref{ch:evaluation} the time taken to run the system is discussed; in Chapter~\ref{ch:results} the findings are presented and Chapter~\ref{ch:conclusions} contains the conclusion.