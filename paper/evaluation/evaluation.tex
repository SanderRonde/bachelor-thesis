\chapter{Evaluation}\label{ch:evaluation}

Due to there being three major stages to the evaluation (preprocessing, training/testing, translating) as explained in section 4, and these three stages having different bottlenecks, not all experiments were run on the same computers. Both the preprocessing and translating stages require the reading of the entire dataset file, making it very RAM intensive. They also only require CPU work, shifting the bottleneck to the CPU after reading the file. Preprocessing also allows for CPU parallelization, making more CPUs a very good thing to have. As such the first and third stages are run on a computer with 1.5TB of ram and 16 Intel Xeon E5--2630v3 CPUs running at 2.40GHz with 32 threads.The training/testing stage however, is a very GPU intensive task. Due to parallelization being highly effective for this task, as explained in section 4, the amount of GPUs is very influential for this task, offering close to linear performance improvements, the second task was ran on a computer with 1TB of ram, 20 Intel Xeon E5--2650v3 CPUs running at 2.30GHz with 40 threads and 16 NVIDIA Tesla K80 GPUs each with 11.5GB of memory. 

Due to the size of the dataset and the limited amount of time, the decision was made to use only 5\% of the dataset for plots/results, as such the percentage that has been used will be 5\% in the following results unless states otherwise. Keep in mind that this means the first 5\% of the file, not 5\% of users. This means that approximately 3 days of data is being used. This causes many users to not have enough actions to pass the 150 actions baseline. Increasing this percentage will not only increase the actions for existing users (causing an exponential increase in work required) but also introduces new users, also leading to an exponential increase in work. 

In order to get an idea of how the percentage of the dataset takes with the time commands take to run, a percentage of 0.1\% has occasionally also been used. Doing preprocessing took 38 seconds for 0.1\% of the data while it took 40m16s for 5\% of the data, both using 10 CPUs. A very rough estimate puts the duration of preprocessing the entire dataset at about 40 hours also using 10 CPUs. Doing the training/testing stage took 1h51m using 16 GPUs on 0.1\% of the dataset, while 1\% took about 10 hours and 5\% of the data took 62h0m36s, giving a rough estimate of 2000 hours for 100\% of the dataset. The anomaly translation part generally only takes roughly 2 and a half hours, not varying much between dataset sizes as all users need to be iterated through regardless and no other heavy CPU work is being done. The biggest time sink for this part is loading the dataset file itself at about 2h15m. Putting this all together, the entire process takes 65h10m52s for 5\% of the data. The 5\% dataset contains 50,276,292 actions, meaning the network can handle about 214 actions per second, making this very fit for real-time anomaly detection. The actual testing stage (without training) takes even shorter, generally only taking a few seconds per user for all their actions, which would make a network that doesn't continue learning after the initial training even more feasible to run.

\begin{figure}
	\begin{center}
		\includegraphics[scale=0.1]{evaluation/all_deviations}
	\end{center}
	\caption{All deviations from the IQR.~\label{fig:iqr_scale}}
\end{figure}

In figure~\ref{fig:iqr_scale}, the deviations from the IQR can be seen for all actions. This deviation is equal to the distance from the IQR calculated by the following formula, where x is equal to that action's loss value:

\( deviation = (x - Q3) / IQR \)

This leads to a deviation for which any value greater than 1.5 is an outlier (as discussed in section 4.5). As can be seen in~\ref{fig:iqr_scale}, there are quite a number of outliers, some of which having outliers that fall far beyond the cutoff value of 1.5. From this we can conclude that at least some anomalies are being found. 

\begin{figure}
	\begin{center}
		\includegraphics[scale=0.1]{evaluation/iqrs}
	\end{center}
	\caption{All IQR values.~\label{fig:iqrs}}
\end{figure}

As can be seen in figure~\ref{fig:iqrs}, the IQRs tend to be fairly close to each other, meaning the mean losses are close to each other as well. This shows that the network is relatively successful at modeling user behavior, as the difference between the expected and actual action calculated by the loss function shows few big spikes. A network that is unsuccessful at this would have inconsistent IQRs as the losses would fluctuate more from user to user and would show higher values indicating bad predictions.

\begin{figure}
	\begin{center}
		\includegraphics[scale=0.1]{evaluation/highest_offender_time_since_last_access}
	\end{center}
	\caption{The top 10 highest offenders' seconds since last access.~\label{fig:time_since_last_access}}
\end{figure}

Focussing on the highest offending users allows us to see more clearly why the network thought certain users were deemed anomalies and whether the network may have been right.

%TODO: maybe not?
Taking a look at the time since the last network access for the top 10 highest offending users in figure~\ref{fig:time_since_last_access} clearly shows that some of these users were being very inconsistent in the time since their last network access. The boxplot has some values that are very far from its average. This could be one of the reasons the network has flagged them as big anomalies.

\begin{figure}
	\begin{center}
		\includegraphics[scale=0.1]{evaluation/highest_offender_percentage_failed_logins}
	\end{center}
	\caption{The top 10 highest offenders' percentages of failed logins.~\label{fig:percentage_failed_logins}}
\end{figure}

%TODO: maybe not?
Figure~\ref{fig:fig:percentage_failed_logins} shows the percentage of failed logins of all actions made by the top 10 highest offending users. This figure also shows some users with percentages that lie very far from the boxplot's average. As this value is a percentage over all their logins it has to rise slowly. This is why there are no sudden jumps and why the outliers are still outliers even if they were made over lots of actions.

